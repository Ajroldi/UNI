\documentclass[12pt,a4paper]{article}
\usepackage[utf8]{inputenc}
\usepackage[italian]{babel}
\usepackage{amsmath}
\usepackage{amssymb}
\usepackage{amsthm}
\usepackage{tikz}
\usepackage{pgfplots}
\usepackage{array}
\usepackage{booktabs}
\usepackage{geometry}
\usepackage{fancyhdr}
\usepackage{xcolor}
\usepackage{tcolorbox}
\usepackage{enumitem}

\geometry{margin=2.5cm}
\pgfplotsset{compat=1.18}

% Definizioni colori
\definecolor{boxbg}{RGB}{240,248,255}
\definecolor{warningbg}{RGB}{255,240,240}

% Box colorati
\newtcolorbox{defbox}{colback=boxbg, colframe=blue!75!black, title=Definizione}
\newtcolorbox{exbox}{colback=green!5, colframe=green!75!black, title=Esempio}
\newtcolorbox{warnbox}{colback=warningbg, colframe=red!75!black, title=Attenzione!}

\title{\textbf{Mini-Libro di Studio}\\Ricerca Operativa}
\author{Appunti delle Lezioni 1-6}
\date{\today}

\begin{document}

\maketitle
\tableofcontents
\newpage

\section{Introduzione alla Ricerca Operativa}

\subsection{Cos'è la Ricerca Operativa?}

\begin{defbox}
La \textbf{Ricerca Operativa (RO)} è la disciplina che studia come prendere decisioni ottimali quando le risorse sono limitate.
\end{defbox}

\begin{exbox}
\textbf{Esempio pratico}: Hai 100€ e devi decidere cosa comprare per massimizzare la tua felicità rispettando il budget.
\end{exbox}

\subsection{Schema del Processo Decisionale}

\begin{center}
\begin{tikzpicture}[node distance=1.5cm]
\node (real) [rectangle, draw, fill=blue!20, text width=3cm, text centered] {Problema Reale};
\node (model) [rectangle, draw, fill=green!20, text width=3cm, text centered, below of=real] {Modello Matematico};
\node (sol) [rectangle, draw, fill=yellow!20, text width=3cm, text centered, below of=model] {Soluzione Matematica};
\node (dec) [rectangle, draw, fill=orange!20, text width=3cm, text centered, below of=sol] {Decisione nel Mondo Reale};

\draw[->, thick] (real) -- node[right] {ASTRAZIONE} (model);
\draw[->, thick] (model) -- node[right] {RISOLUZIONE} (sol);
\draw[->, thick] (sol) -- node[right] {INTERPRETAZIONE} (dec);
\end{tikzpicture}
\end{center}

\subsection{I 3 Pilastri di Ogni Modello}

\begin{center}
\fbox{\textbf{DECISIONI} $\to$ Variabili}\\[0.5em]
\fbox{\textbf{REGOLE} $\to$ Vincoli}\\[0.5em]
\fbox{\textbf{OBIETTIVI} $\to$ Funzione Obiettivo}
\end{center}

\section{Componenti di un Modello}

\subsection{Struttura Generale}

Un problema di ottimizzazione ha sempre questa forma:

\begin{equation}
\begin{aligned}
\max \text{ (o } \min\text{)} \quad & f(x_1, x_2, \ldots, x_n) \\
\text{soggetto a:} \quad & g_1(x_1, \ldots, x_n) \leq b_1 \\
& g_2(x_1, \ldots, x_n) \leq b_2 \\
& \vdots \\
& x_i \geq 0 \text{ per ogni } i
\end{aligned}
\end{equation}

Dove:
\begin{itemize}
\item $f$ = \textbf{funzione obiettivo} (cosa vogliamo ottimizzare)
\item $g_i$ = \textbf{vincoli} (limiti che dobbiamo rispettare)
\item $x_i$ = \textbf{variabili decisionali} (cosa possiamo decidere)
\end{itemize}

\subsection{Esempio: Problema dei Telefonini}

\begin{exbox}
\textbf{Situazione}: Abbiamo pezzi colorati per costruire 2 tipi di telefoni.

\textbf{Dati}:
\begin{itemize}
\item Telefono A: vale 5 punti, usa 2 rossi + 1 giallo
\item Telefono B: vale 8 punti, usa 3 rossi + 1 giallo
\item Disponibili: 9 pezzi rossi, 4 pezzi gialli
\end{itemize}
\end{exbox}

\textbf{Modello matematico}:

\begin{equation}
\begin{aligned}
\max \quad & 5x_A + 8x_B \\
\text{s.t.} \quad & 2x_A + 3x_B \leq 9 \quad \text{(pezzi rossi)} \\
& x_A + x_B \leq 4 \quad \text{(pezzi gialli)} \\
& x_A, x_B \geq 0
\end{aligned}
\end{equation}

\textbf{Rappresentazione grafica}:

\begin{center}
\begin{tikzpicture}[scale=1.2]
\begin{axis}[
    axis lines=middle,
    xlabel={$x_A$},
    ylabel={$x_B$},
    xmin=0, xmax=5,
    ymin=0, ymax=5,
    xtick={0,1,2,3,4,5},
    ytick={0,1,2,3,4,5},
    grid=major,
    width=10cm,
    height=8cm
]

% Vincolo 1: 2xA + 3xB <= 9
\addplot[blue, thick, domain=0:4.5] {(9-2*x)/3};
\addplot[blue, fill=blue!20, opacity=0.3] coordinates {(0,0) (0,3) (4.5,0) (0,0)};

% Vincolo 2: xA + xB <= 4
\addplot[red, thick, domain=0:4] {4-x};

% Regione ammissibile
\addplot[green!50!black, fill=green!20, opacity=0.5] coordinates {(0,0) (0,3) (1,3) (3,1) (4,0) (0,0)};

% Soluzione ottima
\addplot[mark=*, mark size=3pt, red] coordinates {(1,3)};
\node at (axis cs:1,3) [above right] {$(1,3)$ ottimo};

% Curve di livello
\addplot[orange, dashed, domain=0:4] {(20-5*x)/8};
\addplot[orange, dashed, domain=0:4] {(29-5*x)/8};

\end{axis}
\end{tikzpicture}
\end{center}

\textbf{Soluzione ottima}: $x_A = 1$, $x_B = 3$ $\Rightarrow$ Valore = 29 punti

\section{Tipologie di Variabili}

\subsection{Variabili Continue Non Negative}

\begin{equation}
x \geq 0, \quad x \in \mathbb{R}
\end{equation}

\begin{exbox}
Quantità di terreno da coltivare (8.5 ettari è ammesso).
\end{exbox}

\subsection{Variabili Continue Libere in Segno}

\begin{equation}
x \in \mathbb{R} \text{ (può essere negativa)}
\end{equation}

\begin{exbox}
Perturbazione dell'orario dei treni:
\begin{itemize}
\item $\pi_s > 0$ $\to$ ritardo
\item $\pi_s < 0$ $\to$ anticipo
\item $\pi_s = 0$ $\to$ nessun cambiamento
\end{itemize}
\end{exbox}

\subsubsection{Trasformazione Importante}

Una variabile libera può essere scritta come:
\begin{equation}
\pi_s = \pi_s^+ - \pi_s^-
\end{equation}
dove $\pi_s^+, \pi_s^- \geq 0$

\begin{center}
\begin{tabular}{ccc}
\toprule
$\pi$ & $\pi^+$ & $\pi^-$ \\
\midrule
+5 & 5 & 0 \\
-3 & 0 & 3 \\
0 & 0 & 0 \\
\bottomrule
\end{tabular}
\end{center}

\subsection{Variabili Binarie}

\begin{equation}
y \in \{0, 1\}
\end{equation}

\textbf{Significato logico}:
\begin{itemize}
\item $y = 1$ $\to$ ``SÌ, lo faccio''
\item $y = 0$ $\to$ ``NO, non lo faccio''
\end{itemize}

\begin{exbox}
\begin{equation}
y_{\text{fabbrica}} = \begin{cases} 1 & \text{se costruisco la fabbrica} \\ 0 & \text{altrimenti} \end{cases}
\end{equation}
\end{exbox}

\subsection{Variabili Intere}

\begin{equation}
x \in \mathbb{Z}, \quad x \geq 0
\end{equation}

\begin{exbox}
Numero di autobus da acquistare (non posso comprare 2.5 autobus!).
\end{exbox}

\section{Tipologie di Vincoli}

\subsection{Vincoli di Disponibilità ($\leq$)}

\begin{defbox}
\textbf{Significato}: Non posso usare più di quanto ho.
\begin{equation}
\text{Uso delle risorse} \leq \text{Disponibilità}
\end{equation}
\end{defbox}

\begin{exbox}
$2x_A + 3x_B \leq 9$ (al massimo 9 pezzi rossi)
\end{exbox}

\subsection{Vincoli di Fabbisogno ($\geq$)}

\begin{defbox}
\textbf{Significato}: Devo garantire un minimo.
\begin{equation}
\text{Quantità fornita} \geq \text{Fabbisogno minimo}
\end{equation}
\end{defbox}

\begin{exbox}
Vincolo calorico nella dieta:
\begin{equation}
300x_{\text{pasta}} + 250x_{\text{riso}} + \ldots \geq 700
\end{equation}
(almeno 700 calorie)
\end{exbox}

\subsection{Vincoli di Conservazione del Flusso ($=$)}

\begin{defbox}
\textbf{Significato}: Ciò che entra = Ciò che esce
\end{defbox}

\begin{exbox}
Rete di oleodotti:

\begin{center}
\begin{tikzpicture}[node distance=3cm]
\node[circle, draw] (1) {Pozzo 1};
\node[circle, draw, below of=1] (2) {Nodo 2};
\node[circle, draw, right of=2] (3) {Raffineria};
\node[circle, draw, below of=2] (4) {Nodo 4};

\draw[->, thick] (1) -- node[left] {$x_{12}$} (2);
\draw[->, thick] (2) -- node[above] {$x_{23}$} (3);
\draw[->, thick] (2) -- node[left] {$x_{24}$} (4);
\end{tikzpicture}
\end{center}

Vincolo al nodo 2:
\begin{equation}
x_{12} = x_{23} + x_{24}
\end{equation}
\end{exbox}

\subsection{Vincoli Logici}

\subsubsection{OR ($\vee$): Almeno uno}
\begin{equation}
x + y \geq 1
\end{equation}
Almeno una delle due variabili deve valere 1.

\subsubsection{AND ($\wedge$): Entrambi}
\begin{equation}
\begin{cases} x \geq 1 \\ y \geq 1 \end{cases}
\end{equation}
Entrambe le variabili devono valere 1.

\subsubsection{Implicazione ($\to$): Se... allora...}
\begin{equation}
x \leq y
\end{equation}
Se $x=1$ allora deve essere $y=1$.

\subsubsection{XOR ($\oplus$): Uno esclude l'altro}
\begin{equation}
x + y = 1
\end{equation}
Esattamente una delle due deve valere 1.

\subsection{Vincoli ``O questo O quello'' con Variabili Continue}

\textbf{Problema}: Voglio $x = 0$ OPPURE $a \leq x \leq b$ (con $a > 0$)

\textbf{Soluzione}: Introduco una variabile binaria $y$

\begin{equation}
\begin{aligned}
a \cdot y &\leq x \leq b \cdot y \\
y &\in \{0, 1\}
\end{aligned}
\end{equation}

\textbf{Come funziona}:
\begin{itemize}
\item Se $y = 0$ $\Rightarrow$ $x = 0$
\item Se $y = 1$ $\Rightarrow$ $a \leq x \leq b$
\end{itemize}

\section{Trasformazioni Fondamentali}

\subsection{Da $\geq$ a $\leq$}

\begin{equation}
a \cdot x \geq b \quad \Leftrightarrow \quad -a \cdot x \leq -b
\end{equation}

\begin{warnbox}
\textbf{Mnemotecnica}: Moltiplico tutto per -1 e inverto il segno di disuguaglianza.
\end{warnbox}

\subsection{Da $\leq$ a $=$ (Variabili Slack)}

\begin{equation}
a \cdot x \leq b \quad \Leftrightarrow \quad a \cdot x + s = b, \quad s \geq 0
\end{equation}

\textbf{Significato di $s$}: Quanto margine ho ancora (risorsa inutilizzata).

\begin{exbox}
Se $2x \leq 10$ e $x = 3$ $\Rightarrow$ $s = 10 - 6 = 4$
\end{exbox}

\subsection{Da $\geq$ a $=$ (Variabili Slack)}

\begin{equation}
a \cdot x \geq b \quad \Leftrightarrow \quad a \cdot x - s = b, \quad s \geq 0
\end{equation}

\textbf{Significato di $s$}: Quanto sto superando il minimo.

\subsection{Da $=$ a due disuguaglianze}

\begin{equation}
a \cdot x = b \quad \Leftrightarrow \quad \begin{cases} a \cdot x \leq b \\ a \cdot x \geq b \end{cases}
\end{equation}

\textbf{Interpretazione geometrica}: Un'uguaglianza è l'intersezione di due semipiani.

\section{Esempi Applicativi Completi}

\subsection{Problema della Dieta}

\begin{exbox}
\textbf{Contesto}: Devo mangiare almeno 700 calorie scegliendo tra vari cibi.

\textbf{Dati}:
\begin{center}
\begin{tabular}{lc}
\toprule
Cibo & Calorie/hg \\
\midrule
Pasta & 300 \\
Riso & 250 \\
Salmone & 200 \\
Pollo & 550 \\
\bottomrule
\end{tabular}
\end{center}

\textbf{Vincolo aggiuntivo}: O salmone O pollo (non entrambi).
\end{exbox}

\textbf{Modello}:

\begin{equation}
\begin{aligned}
\min \quad & \text{(costo totale)} \\
\text{s.t.} \quad & 300x_p + 250x_r + 200x_s + 550x_{po} \geq 700 \\
& x_s \leq M \cdot y_s \\
& x_{po} \leq M \cdot y_{po} \\
& y_s + y_{po} \leq 1 \\
& x_p, x_r, x_s, x_{po} \geq 0 \\
& y_s, y_{po} \in \{0,1\}
\end{aligned}
\end{equation}

\textbf{Spiegazione}:
\begin{itemize}
\item $y_s = 1$ se prendo salmone, 0 altrimenti
\item $y_{po} = 1$ se prendo pollo, 0 altrimenti
\item $y_s + y_{po} \leq 1$ impone la scelta esclusiva
\end{itemize}

\subsection{Panetteria Multi-Periodo}

\begin{exbox}
\textbf{Contesto}: Produco pane per 3 giorni con domanda variabile.

\textbf{Dati}:
\begin{itemize}
\item Domanda: $d_1 = 100$, $d_2 = 150$, $d_3 = 80$
\item Capacità produttiva: $[q, Q] = [50, 200]$ pezzi/giorno
\item Costo produzione: $c_t = 2$€/pezzo
\item Costo inventario: $g = 0.5$€/pezzo/giorno
\item Costo setup: $S = 50$€ (se produco)
\end{itemize}
\end{exbox}

\textbf{Variabili}:
\begin{itemize}
\item $x_t$ = quantità prodotta al giorno $t$
\item $z_t$ = inventario a fine giornata $t$
\item $y_t = 1$ se produco al giorno $t$
\end{itemize}

\textbf{Modello}:

\begin{equation}
\begin{aligned}
\min \quad & \sum_{t=1}^3 (c_t \cdot x_t + g \cdot z_t + S \cdot y_t) \\
\text{s.t.} \quad & z_t = z_{t-1} + x_t - d_t \quad \forall t \\
& q \cdot y_t \leq x_t \leq Q \cdot y_t \quad \forall t \\
& z_0 = 0, \quad z_t \geq 0 \quad \forall t \\
& y_t \in \{0,1\} \quad \forall t
\end{aligned}
\end{equation}

\section{Errori Comuni da Evitare}

\subsection{Errore 1: Moltiplicare variabili tra loro}

\begin{warnbox}
\textbf{SBAGLIATO}:
\begin{equation}
x_{\text{salmone}} \cdot y_{\text{salmone}} \leq 100
\end{equation}

\textbf{Problema}: Questo crea un vincolo \textbf{non lineare} molto difficile da risolvere!

\textbf{CORRETTO}:
\begin{equation}
x_{\text{salmone}} \leq M \cdot y_{\text{salmone}}
\end{equation}
\end{warnbox}

\subsection{Errore 2: Dimenticare i vincoli di collegamento}

\begin{warnbox}
Se introduci una variabile binaria $y$ per controllare una variabile continua $x$, DEVI collegarle!

\textbf{SBAGLIATO}:
\begin{equation}
\begin{aligned}
& x_s + x_{po} \leq 100 \\
& y_s + y_{po} \leq 1 \\
& \text{(mancano i collegamenti!)}
\end{aligned}
\end{equation}

\textbf{CORRETTO}:
\begin{equation}
\begin{aligned}
& x_s + x_{po} \leq 100 \\
& x_s \leq M \cdot y_s \\
& x_{po} \leq M \cdot y_{po} \\
& y_s + y_{po} \leq 1
\end{aligned}
\end{equation}
\end{warnbox}

\subsection{Errore 3: Big-M troppo grande}

\begin{warnbox}
Se $M$ è troppo grande (es. $M = 10^{10}$), il solver lavora male.

\textbf{REGOLA}: Usa il $M$ valido più piccolo possibile!

\textbf{Esempio}: Se $x \leq 100$ sempre, usa $M = 100$, non $M = 999999$.
\end{warnbox}

\subsection{Errore 4: Usare $=$ quando serve $\leq$}

\begin{warnbox}
\textbf{SBAGLIATO} (nel set covering):
\begin{equation}
\sum_{j \in F_i} x_j = 1 \quad \text{(troppo rigido!)}
\end{equation}

\textbf{CORRETTO}:
\begin{equation}
\sum_{j \in F_i} x_j \geq 1 \quad \text{(almeno uno)}
\end{equation}
\end{warnbox}

\subsection{Errore 5: Confondere obiettivi multipli}

\begin{warnbox}
Se hai 2 obiettivi:
\begin{enumerate}
\item Minimizzare costi
\item Massimizzare copertura
\end{enumerate}

\textbf{NON puoi scrivere}:
\begin{equation}
\begin{aligned}
\min & \sum c_j y_j \\
\max & \sum x_{ij}
\end{aligned}
\end{equation}

\textbf{Devi scegliere}:
\begin{itemize}
\item \textbf{Opzione A}: Funzione pesata $\to$ $\max \alpha \sum x_{ij} - \sum c_j y_j$
\item \textbf{Opzione B}: Vincolo su uno $\to$ $\min \sum c_j y_j$ con $\sum x_{ij} \geq \text{target}$
\end{itemize}
\end{warnbox}

\section{Formulario Rapido}

\subsection{Trasformazioni Base}

\begin{center}
\begin{tabular}{rcl}
\toprule
$\max f(x)$ & $\Leftrightarrow$ & $\min -f(x)$ \\[0.5em]
$ax \geq b$ & $\Leftrightarrow$ & $-ax \leq -b$ \\[0.5em]
$ax \leq b$ & $\Leftrightarrow$ & $ax + s = b, \; s \geq 0$ \\[0.5em]
$ax = b$ & $\Leftrightarrow$ & $\begin{cases} ax \leq b \\ ax \geq b \end{cases}$ \\
\bottomrule
\end{tabular}
\end{center}

\subsection{Vincoli Logici (variabili binarie)}

\begin{center}
\begin{tabular}{ll}
\toprule
\textbf{Tipo} & \textbf{Formulazione} \\
\midrule
OR & $x + y \geq 1$ \\
AND & $x \geq 1, \; y \geq 1$ \\
Implicazione & $x \leq y$ \\
XOR & $x + y = 1$ \\
\bottomrule
\end{tabular}
\end{center}

\subsection{Linking Constraints}

\begin{equation}
x \leq M \cdot y, \quad y \in \{0,1\}
\end{equation}

\textbf{Effetto}: Se $y=0$ $\Rightarrow$ $x=0$; se $y=1$ $\Rightarrow$ $x \leq M$

\end{document}